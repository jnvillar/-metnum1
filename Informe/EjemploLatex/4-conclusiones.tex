\section{Conclusiones}

  Además de exponer otra de las tantas posibilidades de aplicación a problemas físicos de la resolución de sistemas de ecuaciones lineales, la realización de este trabajo permitió, a partir de las pruebas experimentales llevadas a cabo, extraer las siguientes conclusiones con respecto a los dos aspectos del problema cubiertos por estas pruebas: el rendimiento de los métodos y el comportamiento del sistema.

  Las comparaciones realizadas entre los métodos estudiados con respecto a su rendimiento temporal muestran que este fue muy similar entre ambos cuando solo se realizó el cálculo para una única instancia. Sin embargo, al realizar el cálculo repetidamente para varias instancias, manteniendo constantes los parámetros del problema pero variando los datos de entrada, el método de Factorización LU se mostró claramente superior al de Eliminación Gaussiana. Dado que este escenario podría presentarse en el caso de aplicación estudiado, por ejemplo, al analizar la evolución temporal del sistema, la utilización del método de Factorización LU podría ser convieniente.

  Cabe destacar también que el aumento de la granularidad de la discretización repercute negativamente en el rendimiento de ambos métodos por igual. Esto indica que debe estudiarse con cuidado la granularidad necesaria, considerando las limitaciones particulares del contexto a la hora de realizar los cálculos.

  En cuanto al comportamiento del sistema, se pudieron verificar las hipótesis de que la cercanía de la isoterma en el interior de la pared del horno es directamente proporcional a la temperatura externa e inversamente proporcional al grosor de la pared del horno. Esto arroja información que puede ser útil a la hora de prevenir situaciones de riesgo en el contexto de aplicación real. Por otra parte, se comprobó que la precisión de la estimación de la posición de la isoterma es más sensible a cambios en la granularidad de radios que de ángulos; no obstante, sería útil continuar experimentando para verificar el efecto que una menor granularidad de ángulos puede tener en el índice de peligrosidad y en la detección de situaciones de riesgo.
